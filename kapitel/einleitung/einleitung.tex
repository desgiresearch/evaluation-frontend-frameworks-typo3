\section{Einleitung}
\subsection{Problemstellung}
Eine moderne Webseite hat heutzutage viele Erwartungen zu erfüllen. Dadurch das die Nutzung des Internets durch mobile Endgeräte stetig steigt\footcite[5]{Cisco.GlobalTraffic} muss eine Webseite für Mobile als auch für Desktop angepasst werden.
Ansätze wie Mobile First oder Mobile Only sind allgemein bekannt.\footcite[Vgl]{Krug.2018}
Hinzu kommt der Fakt, dass die Anzahl der Webseiten weltweit kontinuierlich steigen.\footcite{InternetLiveStats}
Die Webseite darf aber nicht in der Menge untergehen und soll bei Suchmaschienen möglichst weit oben erscheinen. 
Außerdem ist Geschwindigkeit ein entscheidender Faktor wie diverse Beispiele zeigen. Pinterest steigerte ihre Neuanmeldungen durch Suchmaschienenergebnisse um 15\% durch eine Reduzierung der Ladezeit von 40\%. \footcite{Pinterest.Increased}.
Mobify als Anbieter von ecommerce Progressive Web Apps gaben in einem Report, dass eine Ladezeitersparnis von 100 Millisekunden die Conversion-Rate um je 1,55\% steigert.\footcite[3]{Mobify.2016}
Eine weitere Anforderung stellt der Inhalt der Webseite dar. Website-Besucher suchen nach Informationen und Unterhaltung, nach Lösungen und einem aktiven Austausch zu Problemen und Produkten.\footcite[Vgl.][25]{Loeffler.2014}
Finden sie diese Inhalte nicht, so springen sie ab.
Final gilt es für den Kunden diese Anforderungen für möglichst geringe Kosten und Zeit durch das Entwicklerteam zu realisieren.
\subsection{Zielsetzung der Arbeit}
Im Rahmen dieser Arbeit soll anhand des Content-Management-Systems TYPO3 beantwortet werden, wie gut diese Anforderungen durch die Einbindung eines Frontend-Frameworks gelöst werden können.
Ziel der Forschung ist es dabei herauszufinden wie ein Frontend-Framework dabei helfen kann die 
\subsection{Methodisches Vorgehen und Aufbau der Arbeit}